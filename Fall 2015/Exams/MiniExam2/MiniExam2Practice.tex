\documentclass{article}[11 pt]    % Specifies the document style.

%
\usepackage{color}
\usepackage{amsmath,amsthm}
\usepackage{graphicx}
\usepackage{amsfonts}

\parskip 9pt


\begin{document}           % End of preamble and beginning of text.
\begin{center}
\textbf{Mini Exam 2 Practice}\\[.1cm]
\end{center}

\begin{enumerate}
\item 3-44 in the textbook.  Formulate the problem assuming the goal is to invest the smallest amount of money in year 1.  You can ignore the constraint that puts an upper limit of $\$$120,000 on the investment in each fund in every year.  Clearly write down the three steps of the linear program: define the decision variables, write down the objective and constraints in terms of the decision variables.
Additional questions:
\begin{itemize}
\item Does the optimal solution change if instead our goal is to make $\$$510,000? Use the Sensitivity Reports
\item If instead of minimizing the total amount invested in all four investments in year 1, she instead wants to minimize the total invested in just Funds A,B,C in year one, does the optimal solution change? Use the Sensitivity Report.
\item Reformulate the problem assuming that her investment in Fund D must be at least twice as large as her investment in Fund A in year one.
\end{itemize}
\item Indiana Jones comes upon a cave with 6 gold pieces!  Unfortunately his backpack only has enough room to fit 4 of them.  Luckily he has a jewelry expert with him who tells him that the six pieces have the following values: $\$$10, $\$$7, $\$$20, $\$$8, $\$$15, $\$$3.  Indiana must decide which pieces to put in his backpack.  Formulate the problem as a linear program.  Argue that we do not need to impose an integrality constraint by formulating the problem as a max revenue flow problem (Same thing as min cost flow but instead of minimizing cost we are maximizing revenue).  How much would Indiana be willing to pay for an add-on to his backpack that would allow him to fit one more gold piece.
\end{enumerate}
\end{document}
