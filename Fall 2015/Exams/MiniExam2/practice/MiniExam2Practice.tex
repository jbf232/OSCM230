\documentclass{article}[11 pt]    % Specifies the document style.

%
\usepackage{color}
\usepackage{amsmath,amsthm}
\usepackage{graphicx}
\usepackage{amsfonts}

\parskip 9pt


\begin{document}           % End of preamble and beginning of text.
\begin{center}
\textbf{Mini Exam 2 Practice}\\[.1cm]
\end{center}

\begin{enumerate}
\item Consider a variation of problem 3-44 in the textbook.  Consider the same investment options as given in the problem and the same initial cash amount, but assume that the goal is simply to maximize the amount of money Monique has at the end of the 5 year time span  You can ignore the constraint that puts an upper limit of $\$$120,000 on the investment in each fund in every year.  Clearly write down the three steps of the linear program: define the decision variables, write down the objective and constraints in terms of the decision variables.
Additional questions:
\begin{itemize}
\item How much more money can Monique end up with if she starts with $\$400,000$. Use the Sensitivity Reports, this means that you must justify your answer using element of the sensitivity report.
\item Will the optimal solutions change if Fund D instead brings in a 50$\%$ return?
\end{itemize}
\item Indiana Jones comes upon a cave with 6 gold pieces!  Unfortunately his backpack only has enough room to fit 4 of them.  Luckily he has a jewelry expert with him who tells him that the six pieces have the following values: $\$$10, $\$$7, $\$$20, $\$$8, $\$$15, $\$$3.  Indiana must decide which pieces to put in his backpack to maximize the worth of his treasure haul.  Formulate the problem as a linear program and solve the LP in Excel. 
\begin{itemize}
\item How much would Indiana be willing to pay for an add-on to his backpack that would allow him to fit one more gold piece?
\item The jewelry expert made a mistake and mis-priced the first item at $\$$10 when it is really worth $\$6.50$.  Does this change the optimal solution?  If so, what is the new optimal solution?
\item How much would the value of the item priced at $\$$3 have to increase in order for it to be included in Indiana's backpack.  (Assume original values for all of the items when answering this question)
\item \textbf{Bonus:} Argue that we do not need to impose an integrality constraint by formulating the problem as a max revenue flow problem (Same thing as min cost flow but instead of having a cost for shipping across an arc we can gain a revenue from using an arc). 
\end{itemize}
\end{enumerate}
\end{document}
